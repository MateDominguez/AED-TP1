\documentclass[10pt,a4paper]{article}

\usepackage[spanish,activeacute,es-tabla]{babel}
\usepackage[utf8]{inputenc}
\usepackage{ifthen}
\usepackage{listings}
\usepackage{dsfont}
\usepackage{subcaption}
\usepackage{amsmath}
\usepackage[strict]{changepage}
\usepackage[top=1cm,bottom=2cm,left=1cm,right=1cm]{geometry}%
\usepackage{color}%
\newcommand{\tocarEspacios}{%
	\addtolength{\leftskip}{3em}%
	\setlength{\parindent}{0em}%
}

% Especificacion de procs

\newcommand{\In}{\textsf{in }}
\newcommand{\Out}{\textsf{out }}
\newcommand{\Inout}{\textsf{inout }}

\newcommand{\encabezadoDeProc}[4]{%
	% Ponemos la palabrita problema en tt
	%  \noindent%
	{\normalfont\bfseries\ttfamily proc}%
	% Ponemos el nombre del problema
	\ %
	{\normalfont\ttfamily #2}%
	\
	% Ponemos los parametros
	(#3)%
	\ifthenelse{\equal{#4}{}}{}{%
		% Por ultimo, va el tipo del resultado
		\ : #4}
}

\newenvironment{proc}[4][res]{%
	
	% El parametro 1 (opcional) es el nombre del resultado
	% El parametro 2 es el nombre del problema
	% El parametro 3 son los parametros
	% El parametro 4 es el tipo del resultado
	% Preambulo del ambiente problema
	% Tenemos que definir los comandos requiere, asegura, modifica y aux
	\newcommand{\requiere}[2][]{%
		{\normalfont\bfseries\ttfamily requiere}%
		\ifthenelse{\equal{##1}{}}{}{\ {\normalfont\ttfamily ##1} :}\ %
		\{\ensuremath{##2}\}%
		{\normalfont\bfseries\,\par}%
	}
	\newcommand{\asegura}[2][]{%
		{\normalfont\bfseries\ttfamily asegura}%
		\ifthenelse{\equal{##1}{}}{}{\ {\normalfont\ttfamily ##1} :}\
		\{\ensuremath{##2}\}%
		{\normalfont\bfseries\,\par}%
	}
	\renewcommand{\aux}[4]{%
		{\normalfont\bfseries\ttfamily aux\ }%
		{\normalfont\ttfamily ##1}%
		\ifthenelse{\equal{##2}{}}{}{\ (##2)}\ : ##3\, = \ensuremath{##4}%
		{\normalfont\bfseries\,;\par}%
	}
	\renewcommand{\pred}[3]{%
		{\normalfont\bfseries\ttfamily pred }%
		{\normalfont\ttfamily ##1}%
		\ifthenelse{\equal{##2}{}}{}{\ (##2) }%
		\{%
		\begin{adjustwidth}{+5em}{}
			\ensuremath{##3}
		\end{adjustwidth}
		\}%
		{\normalfont\bfseries\,\par}%
	}
	
	\newcommand{\res}{#1}
	\vspace{1ex}
	\noindent
	\encabezadoDeProc{#1}{#2}{#3}{#4}
	% Abrimos la llave
	\par%
	\tocarEspacios
}
{
	% Cerramos la llave
	\vspace{1ex}
}

\newcommand{\aux}[4]{%
	{\normalfont\bfseries\ttfamily\noindent aux\ }%
	{\normalfont\ttfamily #1}%
	\ifthenelse{\equal{#2}{}}{}{\ (#2)}\ : #3\, = \ensuremath{#4}%
	{\normalfont\bfseries\,;\par}%
}

\newcommand{\pred}[3]{%
	{\normalfont\bfseries\ttfamily\noindent pred }%
	{\normalfont\ttfamily #1}%
	\ifthenelse{\equal{#2}{}}{}{\ (#2) }%
	\{%
	\begin{adjustwidth}{+2em}{}
		\ensuremath{#3}
	\end{adjustwidth}
	\}%
	{\normalfont\bfseries\,\par}%
}

% Tipos

\newcommand{\nat}{\ensuremath{\mathds{N}}}
\newcommand{\ent}{\ensuremath{\mathds{Z}}}
\newcommand{\float}{\ensuremath{\mathds{R}}}
\newcommand{\bool}{\ensuremath{\mathsf{Bool}}}
\newcommand{\cha}{\ensuremath{\mathsf{Char}}}
\newcommand{\str}{\ensuremath{\mathsf{String}}}

% Logica

\newcommand{\True}{\ensuremath{\mathrm{true}}}
\newcommand{\False}{\ensuremath{\mathrm{false}}}
\newcommand{\Then}{\ensuremath{\rightarrow}}
\newcommand{\Iff}{\ensuremath{\leftrightarrow}}
\newcommand{\implica}{\ensuremath{\longrightarrow}}
\newcommand{\IfThenElse}[3]{\ensuremath{\mathsf{if}\ #1\ \mathsf{then}\ #2\ \mathsf{else}\ #3\ \mathsf{fi}}}
\newcommand{\yLuego}{\land _L}
\newcommand{\oLuego}{\lor _L}
\newcommand{\implicaLuego}{\implica _L}

\newcommand{\cuantificador}[5]{%
	\ensuremath{(#2 #3: #4)\ (%
		\ifthenelse{\equal{#1}{unalinea}}{
			#5
		}{
			$ % exiting math mode
			\begin{adjustwidth}{+2em}{}
				$#5$%
			\end{adjustwidth}%
			$ % entering math mode
		}
		)}
}

\newcommand{\existe}[4][]{%
	\cuantificador{#1}{\exists}{#2}{#3}{#4}
}
\newcommand{\paraTodo}[4][]{%
	\cuantificador{#1}{\forall}{#2}{#3}{#4}
}

%listas

\newcommand{\TLista}[1]{\ensuremath{seq \langle #1\rangle}}
\newcommand{\lvacia}{\ensuremath{[\ ]}}
\newcommand{\lv}{\ensuremath{[\ ]}}
\newcommand{\longitud}[1]{\ensuremath{|#1|}}
\newcommand{\cons}[1]{\ensuremath{\mathsf{addFirst}}(#1)}
\newcommand{\indice}[1]{\ensuremath{\mathsf{indice}}(#1)}
\newcommand{\conc}[1]{\ensuremath{\mathsf{concat}}(#1)}
\newcommand{\cab}[1]{\ensuremath{\mathsf{head}}(#1)}
\newcommand{\cola}[1]{\ensuremath{\mathsf{tail}}(#1)}
\newcommand{\sub}[1]{\ensuremath{\mathsf{subseq}}(#1)}
\newcommand{\en}[1]{\ensuremath{\mathsf{en}}(#1)}
\newcommand{\cuenta}[2]{\mathsf{cuenta}\ensuremath{(#1, #2)}}
\newcommand{\suma}[1]{\mathsf{suma}(#1)}
\newcommand{\twodots}{\ensuremath{\mathrm{..}}}
\newcommand{\masmas}{\ensuremath{++}}
\newcommand{\matriz}[1]{\TLista{\TLista{#1}}}
\newcommand{\seqchar}{\TLista{\cha}}

\renewcommand{\lstlistingname}{Código}
\lstset{% general command to set parameter(s)
	language=Java,
	morekeywords={endif, endwhile, skip},
	basewidth={0.47em,0.40em},
	columns=fixed, fontadjust, resetmargins, xrightmargin=5pt, xleftmargin=15pt,
	flexiblecolumns=false, tabsize=4, breaklines, breakatwhitespace=false, extendedchars=true,
	numbers=left, numberstyle=\tiny, stepnumber=1, numbersep=9pt,
	frame=l, framesep=3pt,
	captionpos=b,
}


\usepackage{caratula} % Version modificada para usar las macros de algo1 de ~> https://github.com/bcardiff/dc-tex


\titulo{Trabajo practico 1}
\subtitulo{Especificacion y Weakest Precondition}

\fecha{\today}

\materia{Algoritmos y Estructura de Datos}
\grupo{Grupo HIBTAYIGAFWKBCHZLZPJ}

\integrante{Dominguez, Mateo Felipe}{923/24}{matedominguez2@gmail.com}
\integrante{Morrone, Valentina}{35/24}{valenmorrone@hotmail.com}
\integrante{Moran, Juana Gala}{119/24}{juanagalamoran.u@gmail.com}
\integrante{Cuiña, Carolina}{874/24}{cuinacarolina43@gmail.com}

% Declaramos donde van a estar las figuras
% No es obligatorio, pero suele ser comodo
\graphicspath{{../static/}}

\begin{document}

\maketitle

% Punto 1
\section{Especificacion}

% Especificacion punto 1.1
\subsection{grandesCiudades}
\begin{proc}{grandesCiudades}{\In ciudades: \TLista{Ciudad}}{\TLista{Ciudad}}
	\requiere{\neg ciudadesRepetidas(ciudades)}
	\asegura{\neg ciudadesRepetidas(res) \wedge (\forall i : \ent)(0 \leq i < |res| \implicaLuego res[i] \in ciudades \wedge res[i][1] > 50.000)}
\end{proc}

\vspace{0.3cm}

\pred{ciudadesRepetidas}{s: \TLista{Ciudad}}{(\forall i : \ent)(0 \leq i < |s| \implicaLuego (\exists j : \ent)(0 \leq j < |s| \wedge i \neq j \yLuego s[i] = s[j]))}

\vspace{0.3cm}

% Especificaion punto 1.2
\subsection{sumaDeHabitantes}
\begin{proc}{sumaDeHabitantes}{\In menoresDeCiudades: \TLista{Ciudad}, \In mayoresDeCiudades: \TLista{Ciudad}}{\TLista{Ciudad}}
	\requiere{\neg ciudadesRepetidas(menoresDeCiudades) \wedge \neg ciudadesRepetidas(mayoresDeCiudades)}
	\requiere{mismasCiudades(menoresDeCiudades, mayoresDeCiudades)}
	\asegura{mismasCiudades(res, menoresDeCiudades)}
	\asegura{\neg ciudadesRepetidas(res)}
	\asegura{(\forall i,j : \ent)(0 \leq i,j < |menoresDeCiudades| \yLuego menoresDeCiudades[i][0] = mayoresDeCiudades[j][0] \implicaLuego (res[i][1] = menoresDeCiudades[i][1] + mayoresDeCiudades[j][1])))}
\end{proc}

\vspace{0.3cm}

\pred{mismasCiudades}{s: \TLista{Ciudad}, t: \TLista{Ciudad}}{(\forall i : \ent)(0 \leq i < |s| \implicaLuego (\exists j : \ent)(0 \leq j < |t| \yLuego s[i][0] = t[j][0]))}

\vspace{0.3cm}

% Especificacion punto 1.3
\subsection{hayCamino}
\begin{proc}{hayCamino}{\In distancias: \TLista{\TLista{\ent}}, \In desde: \ent, \In hasta: \ent}{Bool}
	\requiere{esCuadrada(distancias)}
	\requiere{(\forall i,j : \ent)(0 \leq i,j < |s| \implicaLuego s[i][j] = s[j][i] \wedge s[i][j] \geq 0)}
	\requiere{0 \leq desde,hasta < |distancias|}
	\asegura{res = True \iff (\exists \ camino : \TLista{\ent})(esCamino(distancias, camino, desde, hasta))}
\end{proc}

\vspace{0.3cm}

\pred{esCuadrada}{A : \TLista{\TLista{\ent}}}{(\forall i : \ent)(0 \leq i < |A| \implicaLuego |A[i]| = |A|)}
\vspace{0.1cm}
\pred{esCamino}{s: \TLista{\TLista{\ent}}, t: \TLista{\ent}, n: \ent, m: \ent}{(2 \leq |t| \leq |s| \yLuego t[0] = n \yLuego t[|t|-1] = m \yLuego (\forall i : \ent)(0 \leq i < |t|-1 \implicaLuego 0 \leq t[i] < |s| \wedge s[t[i]][t[i+1]] > 0))}

\vspace{0.3cm}

% Especificaion punto 1.4
\subsection{cantidadCaminosNSaltos}
\begin{proc}{cantidadCaminosNSaltos}{\Inout conexion: \TLista{\TLista{\ent}}, \In n: \ent}{}
	\requiere{conexion = C_0}
	\requiere{esCuadrada(C_0)}
	\requiere{(\forall i,j : \ent)(0 \leq i,j < |C_0|\implicaLuego(C_0[i][j] = C_0[j][i]\wedge(C_0[i][j] = 0 \vee C_0[i][j] = 1)))}
	\requiere{(\forall i,j : \ent)(0 \leq i,j < |C_0| \wedge i = j \implicaLuego C_0[i][j] = 0)}
	\requiere{n \geq 1}
	\asegura{|conexion| = |C_0|}
	\asegura{(\forall i,j : \ent)(0 \leq i,j < |conexion| \implicaLuego conexion[i][j] = conexion[j][i])}
	\asegura{(\exists t : \TLista{\TLista{\TLista{\ent}}})(|t| = n \wedge t[0] = C_0 \yLuego (\forall i,j : \ent)(0 \leq i < |t| - 1 \wedge 0 \leq j < |t| \implicaLuego |t[i]|=|t[j]| \wedge \\ esCuadrada(t[j]) \wedge APorBEsC(t[i], C_0, t[i+1])) \wedge conexion = t[n-1])}
\end{proc}

\vspace{0.3cm}

\pred{APorBEsC}{A : \TLista{\TLista{\ent}}, B : \TLista{\TLista{\ent}}, C : \TLista{\TLista{\ent}}}{(\forall i,j : \ent)(0 \leq i,j < |C| \implicaLuego (C[i][j] = \sum\limits_{k=0}^{|C| - 1} A[i][k]*B[k][j]))}

\vspace{0.3cm}

% Especificacion punto 1.5
\subsection{caminoMinimo}
\begin{proc}{caminoMinimo}{\In origen: \ent, \In destino: \ent, \In distancias: \TLista{\TLista{\ent}}}{\TLista{\ent}}
	\requiere{esCuadrada(distancias)}
	\requiere{(\forall i,j : \ent)(0 \leq i,j < |s| \implicaLuego s[i][j] = s[j][i] \wedge s[i][j] \geq 0)}
	\requiere{0 \leq origen,destino < |distancias|}
	\asegura{res = \ensuremath{\langle\rangle} \iff \neg (\exists \ camino : \TLista{\ent})(esCamino(distancias, camino, origen, hasta))}
	\asegura{esCamino(distancias, res, origen, destino) \wedge esMinimo(distancias, res, origen, destino)}
\end{proc}

\vspace{0.3cm}

\pred{esMinimo}{s: \TLista{\TLista{\ent}}, t: \TLista{\ent}, n: \ent, m: \ent}{(\forall w: \TLista{\ent})(esCamino(s, w, n, m) \implicaLuego sumaDistancias(s,t) \leq sumaDistancias(s,w))}
\aux{sumaDistancias}{s: \TLista{\TLista{\ent}}, t: \TLista{\ent}}{\ent}{\sum\limits_{i=0}^{|t| - 2} s[t[i]][t[i+1]]}

\vspace{0.3cm}

% Punto 2
\section{Demostraciones de correctitud}

% Demostracion punto 2.1
\subsection{Demostrar que la implementacion es correcta con respecto a la especificacion}
\begin{lstlisting}[label=implementacion-punto2]
	res = 0
	i = 0
	while (i < ciudades.length) do
		res = res + ciudades[i].habitantes
		i = i + 1
	endwhile
\end{lstlisting}

\vspace{0.3cm}

\textbf{Demostrar \{P\}S\{Q\}}

\vspace{0.1cm}

\noindent Supongamos ciudades = [(a,10),(b,15)]

\begin{table}[h!]
	\begin{tabular}{| l | l | l |} 
		\hline
		Iteracion & i & res  \\ [0.5ex] 
		\hline
		0 & 0 & 0 \\ 
		1 & 1 & 10 \\
		2 & 2 & 10+15=25 \\
		\hline
	\end{tabular}
	\captionsetup{singlelinecheck=off}
	\caption{Tabla de iteracion}
	\label{tab:iteracion}
\end{table}

$ I \equiv 0 \leq i \leq |ciudades| \wedge res = \sum\limits_{j=0}^{i-1} ciudades[j].habitantes$

$ P_c \equiv \{res = 0 \wedge i = 0\}$

$ Q_c \equiv \{res = \sum\limits_{i=0}^{|c|-1} ciudades[i].habitantes\}$

$ B = i < |ciudades|$

\vspace{0.3cm}

% Paso 1
\textbf{Paso 1}

\vspace{0.1cm}

\noindent$P_c \implies I \ ? \\ res = 0 \wedge i = 0 \implies 0 \leq i < |ciudades| \wedge res = \sum\limits_{j=0}^{i-1} ciudades[j].habitantes$

Analizamos variable a variable: \\ $\bullet \ i = 0 \implies 0 \leq i < |ciudades| \\ \bullet \ res = \sum\limits_{j=0}^{i-1} ciudades[j].habitantes = \sum\limits_{j=0}^{-1} ciudades[j].habitantes = 0 \implies res = 0 $

\vspace{0.3cm}

% Paso 2
\textbf{Paso 2}

\vspace{0.1cm}

\noindent$\{I \wedge B\} S \{I\} \\ \{I \wedge B\} \implies wp(S,I) \ ?$

\noindent$\bullet \ \{I \wedge B\} = \{0 \leq i < |ciudades| \wedge \sum\limits_{j=0}^{i-1} ciudades[j].habitantes\} \\ \bullet \ wp(S_1;S_2,I) = wp(S_1,wp(S_2,I))$

\begin{enumerate}
	\item $wp(S_2,I) = wp(i := i + i, 0 \leq i \leq |ciudades| \wedge res = \sum\limits_{j=0}^{i-1} ciudades[j].habitantes) \equiv \\ def(i+1) \yLuego 0 \leq i+1 \leq |ciudades| \yLuego res = \sum\limits_{j=0}^{i} ciudades[j].habitantes \equiv \\ 0 \leq i+1 \leq |ciudades| \wedge res = \sum\limits_{j=0}^{i} ciudades[j].habitantes$ 
	\item $wp(S_1,wp(S_2,I)) \equiv wp(res := res + ciudades[i].habitantes, 0 \leq i+1 \leq |ciudades| \wedge res = \sum\limits_{j=0}^{i} ciudades[j].habitantes) \equiv \\ def(ciudades) \wedge def(ciudades[i].habitantes) \yLuego 0 \leq i+1 \leq |ciudades| \yLuego res + ciudades[i].habitantes= \sum\limits_{j=0}^{i} ciudades[j].habitantes \equiv \\ 0 \leq i < |ciudades| \wedge 0 \leq i+1 \leq |ciudades| \wedge res + ciudades[i].habitantes= \sum\limits_{j=0}^{i} ciudades[j].habitantes \equiv \\ 0 \leq i \leq |ciudades| - 1 \wedge res + ciudades[i].habitantes = \sum\limits_{j=0}^{i} ciudades[j].habitantes \equiv \\ 0 \leq i \leq |ciudades| - 1 \wedge res = \sum\limits_{j=0}^{i} ciudades[j].habitantes - ciudades[i].habitantes \equiv \\ 0 \leq i \leq |ciudades| - 1 \wedge res = \sum\limits_{j=0}^{i-1} ciudades[j].habitantes$
\end{enumerate}

\vspace{0.3cm}

% Paso 3
\textbf{Paso 3}

\vspace{0.1cm}

\noindent$I \wedge \neg B \implies Q \ ?$

\begin{enumerate}
	\item $(0 \leq i < |ciudades| \wedge res = \sum\limits_{j=0}^{i-1} ciudades[j].habitantes \wedge i \geq |ciudades|) \implies res = \sum\limits_{j=0}^{|c|-1} ciudades[j].habitantes \equiv \\ (i = |ciudades| \wedge res = \sum\limits_{j=0}^{i-1} ciudades[j].habitantes) \implies res = \sum\limits_{j=0}^{|c|-1} ciudades[j].habitantes \equiv \\ res = \sum\limits_{j=0}^{|c|-1} ciudades[j].habitantes \implies res = \sum\limits_{j=0}^{|c|-1} ciudades[j].habitantes$
\end{enumerate}

\vspace{0.3cm}

%teorema de terminacion
\textbf{Teorema de terminacion}

\vspace{0.1cm}

\noindent$ f_v = |ciudades| - i $

\vspace{0.3cm}

\textbf{Paso 1}

\vspace{0.1cm}

\noindent$\{I \wedge B \wedge f_v = v_0\}S\{f_v < v_0\}$

\vspace{0.1cm}

\noindent$ Verificamos \ \forall \ v_0, \ wp(S,f_v < v_0)$

\vspace{0.1cm}

\begin{enumerate}
	\item $wp(S_1,wp(S_2,f_v < v_0)) \equiv \\ wp(res := res + ciudades[i].habitantes,wp(i := i + 1,|ciudades| - i < V_0)) \equiv \\ wp(res := res + ciudades[i].habitantes, def (i) \wedge def(1)) \yLuego |ciudades|-(i + 1) < V_0 \equiv \\ wp(res := res + ciudades[i].habitantes,|ciudades|- i - 1 <  V_0) \equiv \\ def(res) \wedge def(ciudades) \wedge 0 \leq i < |ciudades| \yLuego |ciudades|- i - 1 < V_0 \equiv \\ 0 \leq i < |ciudades| \yLuego |ciudades|- i - 1 < |ciudades| \equiv True $ 
\end{enumerate}

\vspace{0.3cm}

\textbf{Paso 2}

\vspace{0.1cm}

\noindent$\{I \wedge f_v < 0\} \implies \neg B \ ?$

\begin{enumerate}
	\item $ 0 \leq i < |ciudades| \wedge res = \sum\limits_{j=0}^{i-1} ciudades.[i].habitantes \wedge |ciudades| - i < 0 \implies i \geq |ciudades| \equiv \\ i < |ciudades| \wedge res = \sum\limits_{j=0}^{i-1} ciudades.[i].habitantes \wedge |ciudades| < i \implies i \geq |ciudades| \equiv \\ i = |ciudades| \wedge res = \sum\limits_{j=0}^{i-1} ciudades.[i].habitantes \implies i \geq |ciudades| \equiv \\ i = |ciudades| \implies i \geq |ciudades| $ 
\end{enumerate}

\pagebreak

% Demostracion paso 2.2
\subsection{Demostrar que el valor devuelto es mayor a 50.000}

\vspace{0.3cm}

\noindent$P_c \equiv wp(S_1, Q_1) \\ S_1 \equiv res:= res + ciudades[i].habitantes \\ Q_1 \equiv wp(S_2, Q_c) \\ S_2 \equiv i:= i + 1 \\ Q_c \equiv res = \sum\limits_{j = 0}^{|c|-1} ciudades[j].habitantes \wedge res > 50{.}000$

\begin{enumerate}
	\item $wp(S_2,Q_c) \equiv wp(i:= i+1, res= \sum\limits_{j = 0}^{|c|-1} ciudades[j].habitantes \wedge res > 50{.}000) \equiv \\ def (i) \wedge def (1) \wedge res= \sum\limits_{j = 0}^{|c|-1} ciudaades[j].habitantes \wedge res > 50{.}000 \equiv \\ res= \sum\limits_{j=0}^{|c|-1} ciudades[j].habitantes \wedge res > 50{.}000$
	\item $wp(S_1,Q_1) \equiv wp(res= res + ciudades[i].habitantes, res=\sum\limits_{j=0}^{|c|-1} ciudades[i].habitantes \wedge res > 50{.}000) \equiv def(i)\wedge def(c) \wedge (o \leq i < |c|) \yLuego res + ciudades[i].habitantes = \sum\limits_{j = 0}^{|c|-1} ciudades[j].habitantes \wedge res + ciudades[i].habitantes > 50{.}000 \equiv \\ (0 \leq i < |c|) \yLuego res + ciudades[i].habitantes = \sum\limits_{j = 0}^{|c|-1} ciudades[j].habitantes \wedge res + ciudades[i].habitantes > 50{.}000$
\end{enumerate}

\noindent$Proponemos:\\ P_c \equiv res = 0 \wedge i = 0 \wedge \sum\limits_{j = 0}^{|c|-1} ciudades[j].habitantes > 50{.}000 \\ \\ Vemos: \\ P_c \implies Q_c \\ (res=0 \wedge i=0 \wedge \sum\limits_{j = 0}^{|c|-1} ciudades[j].habitantes > 50{.}000)\implies res=\sum\limits_{j = 0}^{|c|-1} ciudades[j].habitantes > 50{.}000 \\ \\ P_c \implies I\\ (res=0 \wedge i=0 \wedge \sum\limits_{j = 0}^{|c|-1} ciudades[j].habitantes > 50{.}000) \implies (0 \leq i \leq |c| \wedge res= \sum\limits_{j = 0}^{i-1} ciudades[j].habitantes)$

\end{document}